\begin{longtable}{ | l | p{6cm} | p{2cm} | p{4cm} |}
\caption{Requisitos Funcionais Hidra}\\
\hline
\textbf{ID} & \textbf{Requisito} & \textbf{RA Cambuci} & \textbf{Solução}  \\
\hline
\endfirsthead
\multicolumn{4}{c}%
{\tablename\ \thetable\ -- \textit{Requisitos Funcionais Hidra}} \\
\hline
\textbf{ID} & \textbf{Requisito} & \textbf{RA Cambuci} & \textbf{Solução}  \\
\hline
\endhead
\hline \multicolumn{4}{r}{\textit{Continua na página seguinte}} \\
\endfoot
\hline
\endlastfoot
	RF-01
	& A biblioteca \textit{Hidra} deve permitir a \textbf{inclusão de ativos de software}, levando em consideração a composição de um ativo por diferentes artefatos.
  & RA-AS[1]
  & Os ativos reusáveis de software são armazenados no repositório em forma de diretórios, por meio, da segunda forma de armazenamento RAS \cite{omg2005}. \\ \hline

	RF-02
  & A biblioteca \textit{Hidra} deve fornecer mecanismos a fim de listar artefatos que compõem um ativo de software armazenado no repositório.
  & RA-AS[1]
  & Requisito implementado por meio dos métodos \textit{Asset.getSolution()} e \textit{Asset.setSolution()}. \\ \hline
   
  RF-03
  & A biblioteca \textit{Hidra} deve possuir uma estrutura padronizada de representação, comunicação e armazenamento de ativos de software. 
  & RA-AS[2] 
  & Foi adotado o padrão RAS atualmente em sua versão 2.2. \\ \hline
	
	RF-04
  & A biblioteca \textit{Hidra} deve garantir que todo novo ativo de software seja validado e certificado de acordo com o padrão adotado.
  & RA-AS[2]
  & As regras especificadas no padrão RAS, expressas em forma de um XSD \textit{NomeArquivoXSD.xsd}, são consultadas ao validar e certificar um Ativo antes de qualquer atualização ou inserção (método \textit{Asset.validate()}).\\ \hline

  RF-05
  & A biblioteca \textit{Hidra} deve garantir que ativos de software, que não forem mais utilizados, possam ser removidos do repositório.
  & RA-AS[3]
  & Requisito implementado por meio do método \textit{Repository.removeAsset(Asset asset)}. \\ \hline
  
  RF-06
  & A biblioteca \textit{Hidra} deve possibilitar a adição de informações para classificação de um ativo e também o contexto de sua utilização.
  & RA-AS[4] 
  & Requisito implementado por meio dos métodos \textit{Asset.getClassification()} e \textit{Asset.setClassification()}. \\ \hline

  RF-07
  & A biblioteca \textit{Hidra} deve possibilitar a adição de informações sobre regras para instalação, personalização, e utilização do ativo.
  & extensão do requisito RA-AS[4] baseando-se no padrão RAS.
  & Requisito implementado por meio dos métodos \textit{Asset.getUsage()} e \textit{Asset.setUsage()}. \\ \hline

  RF-08
  & A biblioteca \textit{Hidra} deve possibilitar o registro de dependência entre ativos.
  & RA-AS[5] 
  & Requistio implementado por meio dos métodos \textit{Asset.getRelatedAssets()} e \textit{Asset.setRelatedAsset()}. \\ \hline

  RF-09
  & A biblioteca \textit{Hidra} deve oferecer informações relevantes a todos os interessados, sobre mudanças que aconteçam no ativo de software: data de alteração, autor da alteração, o que foi alterado e descrição sobre a alteração.
  & RA-AS[6] 
  & Requisito implementado por meio do método \textit{Asset.getLog()} (Remover do texto: Obs.: operação do git equivalente: git log arquivo.) \\ \hline

  RF-10
  & A biblioteca \textit{Hidra} deve fornecer mecanismos a fim de listar ativos armazenados no repositório.
  & RA-AS[7]
  & Requisito implementado por meio do método \textit{Repository.listAssets()} \\ \hline

  RF-11
  & A biblioteca \textit{Hidra} deve fornecer mecanismos a fim de recuperar um ativo armazenado no repositório (download).
  & RA-AS[7] 
  & Requisito implementado por meio do método \textit{Repository.retrieveAsset()} \\ \hline

  RF-12
  & A biblioteca \textit{Hidra} deve fornecer mecanismos a fim de buscar ativos armazenados no repositório.
  & RA-AS[7] 
  & Requisito não implementado na versão atual da biblioteca \textit{Hidra} (Trabalho Futuro). \\ \hline
%
% continuar a partir daqui.
%
  RA-AS[17] 
  & A arquitetura de referência deve possibilitar que  repositórios de ativos de software realizem o  gerenciamento de transação, garantindo a atomicidade,  consistência, isolamento e durabilidade.
  & A biblioteca \textit{Hidra} deve fornecer mecanismos que garantem a atomicidade, consistência e isolamento de transações de controle de ativos de software. \\ \hline
  RAS[1] e RAS[2] 
  & A arquitetura de referência de possibilitar que repositórios de  ativos de software desenvolvidos para persistir diferentes tipos  de ativos possam ser facilmente integrados.

A arquitetura de referência deve possibilitar que repositórios de ativos de software implementados em linguagens de  programação distintas e sob diferentes plataformas possam ser  facilmente integrados.
    &  A biblioteca \textit{Hidra} de software deve fornecer mecanismos de integração que permitem a persistencia de diferentes tipos de ativos implementados em diferentes linguagens de programação.
 \\ \hline
 
  RAS[3] 
  & A arquitetura de referência deve prover mecanismos para que  repositórios de ativos de software na forma de serviços possam  ser publicados e posteriormente descobertos por aplicações  cliente.
  & A biblioteca \textit{Hidra} deve prover mecanismos para que suas funcionalidades sejam executadas na forma de serviços, que serão publicados e posteriormente descobertos por aplicações clientes.

 \\ \hline
  RAS[4] & 
A arquitetura de referência de prover mecanismos para que  repositórios de ativos de software orientados a serviço possam  ser compostos por processos de negócio ou utilizados por  aplicações cliente. & Requisitos não-funcionais 1: A biblioteca \textit{Hidra} de software deve permitir acesso externo de maneira automatizada
2: A biblioteca \textit{Hidra} de software deve permitir que serviços sejam usados por meio de orquestração (Camada de webservice permitirá isso).


 \\ \hline
  RAS[5] & 
A arquitetura de referência deve viabilizar o desenvolvimento  de repositórios de ativos de software que disponibilizem  informações sobre suas características e direções normativas de  uso, por meio de descrições padronizadas.
 & A biblioteca \textit{Hidra} 
deve garantir que o desenvolvimento de repositórios de ativos informem suas características  e direções normativas de uso por meio de descrições padronizadas de suas funcionalidades.

 \\ \hline
RAS[6] & 
A arquitetura de referência deve viabilizar o desenvolvimento  de repositório de ativos de software que disponibilizem  descrições semânticas, permitindo assim sua classificação nos  repositórios de serviço. & O escopo inicial do desenvolvimento da biblioteca hidra tem como foco os requisitos fundamentais de repositorio de ativos de software, transportando o requisito RAS[6] para uma abordagem futura em uma nova análise de escopo
\\ \hline
RAS[7] & 
A arquitetura de referência deve viabilizar o desenvolvimento  de repositório de ativos de software que tenham à disposição  informações e documentos relacionados às suas características  de qualidade. & O escopo inicial do desenvolvimento da biblioteca hidra tem como foco os requisitos fundamentais de repositorio de ativos de software, transportando o requisito RAS[7] para uma abordagem futura em uma nova análise de escopo.
 \\ \hline 

RAS[8] & 
A arquitetura de referência deve prover mecanismos para a  captura, monitoramento, registro e sinalização do não  cumprimento de requisitos de qualidade estabelecidos entre  serviços provedores e serviços clientes. & O escopo inicial do desenvolvimento da biblioteca hidra tem como foco os requisitos fundamentais de repositorio de ativos de software, transportando o requisito RAS[8] para uma abordagem futura em uma nova análise de escopo.
 \\ \hline 

RAS[9] & 
A arquitetura de referência deve viabilizar o desenvolvimento
de repositório de ativos de software escalável, capaz de evoluir 
de maneira incremental, por meio da composição de novas 
funcionalidades disponíveis na forma de serviços. & A biblioteca \textit{Hidra} deve prover mecanismos a fim de permitir a adição de novas funcionalidades a biblioteca \textit{Hidra}, por meio de serviços de serivços.
 \\ \hline 
 RAS[10] & 
A arquitetura de referência deve possibilitar que serviços de  repositório de ativos de software e composições desses  serviços sejam tratados uniformemente, ou seja, possam ser  publicados, localizados e utilizados da mesma forma. & A biblioteca de serviços deve prover mecanismos que permitam a seus serviços serem publicados, localizados e utilizados da mesma forma.
 \\ \hline 
 
 RAS[11] & 

A arquitetura de referência deve possibilitar que serviços do  repositório de ativos de software possam interagir diretamente  ou por meio do uso de barramentos de serviço. & 
A biblioteca de controle deve ter uma camada de abstração que permita a integração entre aplicativos, ou que se comuniquem diretamente.

 \\ \hline 

\end{longtable}
