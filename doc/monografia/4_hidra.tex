\chapter{Hidra} \label{chapter:hidra}

Texto.

\section{Requisitos Funcionais} \label{section:sec1}

A abordagem para elaboção dos requisitos funcionais da biblioteca \textit{ Hidra} utiliza-se de derivações do requisitos funcionais contidos na arquitetura de referência Cambuci-LPN (Seção: Trabalhos Relacionados.) em novos requisitos voltados diretamente para a definição da biblioteca \textit{ Hidra}.

A tabela a seguir representa os requisitos funcionais originais da arquitetura de referência Cambuci-LPN, juntamente com seus respectivos identificadores. Os novos requisitos funcionais pertencentes a biblioteca \textit{ Hidra} estão contidos na coluna referênciada por "Requisito Derivado"


\newpage
\begin{longtable}{ | l | p{6cm} | p{6cm} |}
\caption{Tabela de Requisitos Hidra}\\
\hline
\textbf{ID} & \textbf{Requisito Original} & \textbf{Requisito Derivado}  \\
\hline
\endfirsthead
\multicolumn{3}{c}%
{\tablename\ \thetable\ -- \textit{Tabela de Requisitos Hidra}} \\
\hline
\textbf{ID} & \textbf{Requisito Original} & \textbf{Requisito Derivado}  \\
\hline
\endhead
\hline \multicolumn{3}{r}{\textit{Continua na página seguinte}} \\
\endfoot
\hline
\endlastfoot
  RA-AS[1]
  & A arquitetura de referência deve possibilitar que repositórios de ativos de software incluam um novo ativo, que pode ser composto por vários artefatos.
  & A biblioteca de controle deve permitir a inclusão de ativos de software levando em consideração a composição de um ativo por diferentes artefatos.

A biblioteca de controle de controle deve fornecer mecanismos a fim de listar artefatos que compõe um ativo de processo. \\ \hline
    
    RA-AS[2] 
    & A arquitetura de referência deve possibilitar que repositórios de ativos de software forneçam mecanismo para aceitação e certificação de ativos.
    & A biblioteca de controle deve ser capaz de possuir uma estrutura de representação de ativos de software, com finalidade de definir um padrão para o controle de ativos.
A biblioteca de controle de ativos deve garantir que todo novo ativo de software seja validado e certificado de acordo com o padrão RAS. \\ \hline

     RA-AS[3]
     & A arquitetura de referência deve possibilitar que repositórios de ativos de software desativem ativos que não serão mais utilizados.
     & A biblioteca de controle deve garantir que ativos de software, que não forem mais utilizados, possam ser removidos do repositório. \\ \hline
     
    RA-AS[4] 
    & A arquitetura de referência deve possibilitar que repositórios de ativos de software permitam a classificação de um ativo e também informar o contexto de sua utilização.
    & A biblioteca de controle de ativos deve possibilitar a adição de informações para classificação de um ativo e também o contexto de sua utilização. \\ \hline

     RA-AS[5] 
     & A arquitetura de referência deve possibilitar que repositórios de ativos de software registrem a dependência entre ativos.
     & A biblioteca de controle deve possibilitar a descrição dos ativos relacionados por meio de atributos pertencentes a estrutura representante do ativo.
 \\ \hline

    RA-AS[6] 
    & A arquitetura de referência deve possibilitar que repositórios de ativos de software notifiquem os interessados sobre mudanças que aconteçam no ativo. 
    & A biblioteca deve oferecer informações relevantes a todos os interessados, sobre mudanças que aconteçam no ativo de software, como por exemplo, data de alteração e autor da alteração.
 \\ \hline
 
    RA-AS[7] 
    & A arquitetura de referência deve possibilitar que  repositórios de ativos de software permitam realizar  buscas e recuperação dos ativos 
    & A busca e recuperação de ativos não será abordada dentro do escopo inicial do desenvolvimento da biblioteca Hidra, podendo ser implementada futuramente. ) \\ \hline
    
    RA-AS[8] 
    & A arquitetura de referência deve possibilitar que  repositórios de ativos de software permitam a  navegação entre ativos 
    & A abordagem inicial do desenvolvimento da biblioteca hidra considera a navegação entre ativos de software pertencente a um escopo futuro, não estando incluso de primeira estancia) \\ \hline
    
    RA-AS[9] 
    & A arquitetura de referência deve possibilitar que  repositórios de ativos de software aceite múltiplas  fontes de origem de ativos, com o objetivo de facilitar  a integração entre equipes e entre repositórios  diferentes.  
    & A biblioteca de controle  deve fornecer mecanismos de controle e validação de ativos de software oriundos de fontes externas, por meio de serviços padronizados de integração.
 \\ \hline
 
    RA-AS[10]
    & A arquitetura de referência deve possibilitar que  repositórios de ativos de software criem e armazenem  múltiplas versões de um mesmo ativo.
    & A biblioteca de controle deve fornecer mecanismos de versionamento aos ativos de software. \\ \hline
    
    RA-AS[11] 
    & A arquitetura de referência deve possibilitar que  repositórios de ativos de software gerencie a  configuração, como por exemplo, a definição dos itens  do ativo que são configuráveis, o controle de  mudanças dos itens do ativo que são configuráveis.
    & A biblioteca de controle deve oferecer mecanismos  para  gerenciamento da configuração de ativos de software.  \\ \hline
    
    RA-AS[12] 
    &A arquitetura de referência deve possibilitar que  repositórios de ativos de software permita o registro de  impressões dos usuários a respeito da versão do ativo  que eles utilizaram. 
    & O escopo inicial do desenvolvimento da biblioteca hidra tem como foco os requisitos fundamentais de repositorio de ativos de software, transportando o requisito RA-AS[12] para uma abordagem futura em uma nova análise de escopo \\ \hline
    RA-AS[13] 
    & A arquitetura de referência deve possibilitar que  repositórios de ativos de software registrem métricas  coletadas sobre a utilização do ativo.
    & Não condiz com contexto da biblioteca Hidra, uma vez que que o foco da implementação que não visa
a elaboração de métricas. \\ \hline
    RA-AS[14] 
    & A arquitetura de referência deve possibilitar que  repositórios de ativos de software ofereçam  informações relativas ao reúso, iniciativas de reúso,  ativos mais usados, etc.
    & O escopo inicial do desenvolvimento da biblioteca hidra tem como foco os requisitos fundamentais de repositorio de ativos de software, transportando o requisito RA-ASS[14] para uma abordagem futura em uma nova análise de escopo. \\ \hline
    
    RA-AS[15] 
    & A arquitetura de referência deve possibilitar que  repositórios de ativos de software permitam o acesso de  acordo com o papel que o usuário assume.
    & A implementação inicial da biblioteca hidra não abrange o escopo de controle de permissão de usuários. \\ \hline
    
    RA-AS[16] & A arquitetura de referência deve possibilitar que  repositórios de ativos de software garantam a  integridade dos ativos, ou seja, que eles não sofram  alterações não autorizadas.
    & A biblioteca de controle deve garantir que o repositório remoto e principal não sofra alterações não autorizadas.
 \\ \hline
    RA-AS[17] 
    & A arquitetura de referência deve possibilitar que  repositórios de ativos de software realizem o  gerenciamento de transação, garantindo a atomicidade,  consistência, isolamento e durabilidade.
    & A biblioteca de controle deve fornecer mecanismos que garantem a atomicidade, consistência e isolamento de transações de controle de ativos de software. \\ \hline
     RAS[1] e RAS[2] 
     & A arquitetura de referência de possibilitar que repositórios de  ativos de software desenvolvidos para persistir diferentes tipos  de ativos possam ser facilmente integrados.

A arquitetura de referência deve possibilitar que repositórios de ativos de software implementados em linguagens de  programação distintas e sob diferentes plataformas possam ser  facilmente integrados.
    &  A biblioteca de controle de ativos de software deve fornecer mecanismos de integração que permitem a persistencia de diferentes tipos de ativos implementados em diferentes linguagens de programação.
 \\ \hline
 
  RAS[3] 
  & A arquitetura de referência deve prover mecanismos para que  repositórios de ativos de software na forma de serviços possam  ser publicados e posteriormente descobertos por aplicações  cliente.
  & A biblioteca de controle de ativos deve prover mecanismos para que suas funcionalidades sejam executadas na forma de serviços, que serão publicados e posteriormente descobertos por aplicações clientes.

 \\ \hline
  RAS[4] & 
A arquitetura de referência de prover mecanismos para que  repositórios de ativos de software orientados a serviço possam  ser compostos por processos de negócio ou utilizados por  aplicações cliente. & Requisitos não-funcionais 1: A biblioteca de controle de ativos de software deve permitir acesso externo de maneira automatizada
2: A biblioteca de controle de ativos de software deve permitir que serviços sejam usados por meio de orquestração (Camada de webservice permitirá isso).


 \\ \hline
  RAS[5] & 
A arquitetura de referência deve viabilizar o desenvolvimento  de repositórios de ativos de software que disponibilizem  informações sobre suas características e direções normativas de  uso, por meio de descrições padronizadas.
 & A biblioteca de controle 
deve garantir que o desenvolvimento de repositórios de ativos informem suas características  e direções normativas de uso por meio de descrições padronizadas de suas funcionalidades.

 \\ \hline
RAS[6] & 
A arquitetura de referência deve viabilizar o desenvolvimento  de repositório de ativos de software que disponibilizem  descrições semânticas, permitindo assim sua classificação nos  repositórios de serviço. & O escopo inicial do desenvolvimento da biblioteca hidra tem como foco os requisitos fundamentais de repositorio de ativos de software, transportando o requisito RAS[6] para uma abordagem futura em uma nova análise de escopo
\\ \hline
RAS[7] & 
A arquitetura de referência deve viabilizar o desenvolvimento  de repositório de ativos de software que tenham à disposição  informações e documentos relacionados às suas características  de qualidade. & O escopo inicial do desenvolvimento da biblioteca hidra tem como foco os requisitos fundamentais de repositorio de ativos de software, transportando o requisito RAS[7] para uma abordagem futura em uma nova análise de escopo.
 \\ \hline 

RAS[8] & 
A arquitetura de referência deve prover mecanismos para a  captura, monitoramento, registro e sinalização do não  cumprimento de requisitos de qualidade estabelecidos entre  serviços provedores e serviços clientes. & O escopo inicial do desenvolvimento da biblioteca hidra tem como foco os requisitos fundamentais de repositorio de ativos de software, transportando o requisito RAS[8] para uma abordagem futura em uma nova análise de escopo.
 \\ \hline 

RAS[9] & 
A arquitetura de referência deve viabilizar o desenvolvimento
de repositório de ativos de software escalável, capaz de evoluir 
de maneira incremental, por meio da composição de novas 
funcionalidades disponíveis na forma de serviços. & A biblioteca de controle deve prover mecanismos a fim de permitir a adição de novas funcionalidades a biblioteca de controle, por meio de serviços de serivços.
 \\ \hline 
 RAS[10] & 
A arquitetura de referência deve possibilitar que serviços de  repositório de ativos de software e composições desses  serviços sejam tratados uniformemente, ou seja, possam ser  publicados, localizados e utilizados da mesma forma. & A biblioteca de serviços deve prover mecanismos que permitam a seus serviços serem publicados, localizados e utilizados da mesma forma.
 \\ \hline 
 
 RAS[11] & 

A arquitetura de referência deve possibilitar que serviços do  repositório de ativos de software possam interagir diretamente  ou por meio do uso de barramentos de serviço. & 
A biblioteca de controle deve ter uma camada de abstração que permita a integração entre aplicativos, ou que se comuniquem diretamente.

 \\ \hline 

\end{longtable}
